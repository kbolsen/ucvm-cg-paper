\section{Community Velocity Models}
\label{sec:cvms}

The UCVM framework supports different seismic velocity models. 

(temp material)

Community velocity models vary widely in their area of coverage, depth extent, and resolution. UCVM is flexible in its support for such variability. However, in order to better accommodate high frequency ground motion simulations, it categorizes models into two general groups: crustal models, and geotechnical layers (GTLs). Crustal models provide subsurface seismic wave velocities associated with basin, crust, and mantle structures. These models may potentially extend to many tens of kilometers below the Earth's surface yet do so at coarse resolutions (TODO: cite CVM-H, CVM-S). Geotechical layers, in contrast, provide velocities for only the near-surface (typically a few hundred meters) at very high resolution (TODO: cite Ely Vs30 GTL). Ground motion simulations, in particular, rely on high-resolution near-surface velocities and therefore a GTL serves to supplement the coarser data provided by crustal models.

TODO: Interpolation of GTL with Crustal

%\textit{
%\color{blue}
%This section will present how CVMs work and the various CVMs available to the community today. It will basically explain that CVMs provide the triplets of Vs, Vp and density, and, as an example, we can expand on a description of CVM-S and CVM-H, including their variations CVM-SI and CVM-H+GTL.
%}

