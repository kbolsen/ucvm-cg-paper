
\section{Additional Utilities and Tools}
\label{sec:utilities}

\subsection{Tiling}

\textcolor{red}{Not being a command, I do not know how we want to describe this feature. I am including it here tentatively because Patrick has at some point suggested it should be included when describing the framework. We could show a schematic figure illustrating how it works and a result.}

\subsection{Small-Scale Heterogeneities Tools}

The last decade has seem a tremendous growth on high performance computing and applications dedicated to earthquake ground motion simulations which is leading the charge toward performing deterministic simulations at high frequencies of engineering interest (\fmax{} 0--10 Hz). With increasing simulation frequencies, seismic velocity models must not only be accurate representations of a region's crustal structure at the geologic scale and represent the geotechnical characteristics of basins and deposits, but should also provide for the adequate representation of the scattering characteristics of the typical heterogeneities observed in geomaterials. To this end, UCVM implements an algorithm generates models that incorporate small-scale heterogeneities to any underlying velocity model. The command \texttt{ssh\_generate} generates a binary float file of heterogeneities on a regular grid space with a normalized amplitude. Then, the command \texttt{ssh\_merge} adds the heterogeneities multiplied by some scaling factor to the model. The introduction of the heterogeneities is actually done by applying the scaled perturbation to the slowness of the underlying mesh velocities, and then converted back to the seismic velocity. The following example generates and merges a set of small scale heterogeneities to ... \textcolor{red}{describe example}. 

\begin{lstlisting}[
	frame=none,
	basewidth={0.45em,0.4em},
	basicstyle=\ttfamily\footnotesize,breaklines=true,
	linewidth=0.98\columnwidth,xleftmargin=0.07\columnwidth,
	numbers=left,numberblanklines=true,numberstyle=\scriptsize\color{mylistingnclr}]
> ./ssh_generate --hurst 0.1 --d1 20 --l1 1000 --st23 5 --seed 3 \
  --n1 1000--n2 2000 --n3 2000 -m ssh.out
> ./ssh_merge --input ssh.out --output combined.mesh \
  --mesh extracted.mesh --std 0.1 --x 2000 --y 2000 --z 1000
\end{lstlisting}

Section \ref{sec:ch-ssh} shows an example of a set of simulations done using models built with UCVM that integrate small-scale heterogeneities, and discusses briefly the effect that this modeling feature has on the ground motion results and their importance.

\subsection{Plotting Utilities}

This section briefly describes a collection of utilities provided with the UCVM platform. These utilities are written in Python scripts that operate as wrappers around the basic single-core commands explained above, and work to produce specific output data commonly used for plotting. Because of brevity, we cannot fully describe here all the options and parameters that each of these commands can take as input and will only illustrate their operability. Additional details can be found at the UCVM User Guide (see Table \ref{tab:manuals}).

\subsubsection{The \textup{\texttt{cross\_section.py}} utility}

This utility plots a cross section of a model given two bounding latitude and longitude points, the depth of the cross section, and a few other parameters. The output of this command is either an image (\textcolor{red}{PNG?}) or a text file (\textcolor{red}{ascii?}) containing the data from the underlying velocity model. It can be run in interactive mode. The following example, for instance, plots a cross section from zero depth down to 10 km at horizontal and vertical grid-sizes of 100 m for the values of \vs{} in the CVM-S model using a discretized color scale between the origin point at coordinates $-34,118$ and the ending point at coordinates $-35.-117$, and outputs the image to the file \texttt{image.png}. The result is shown in Figure \ref{fig:cross.section}

\begin{lstlisting}[
	frame=none,
	basewidth={0.45em,0.4em},
	basicstyle=\ttfamily\footnotesize,breaklines=true,
	linewidth=0.98\columnwidth,xleftmargin=0.07\columnwidth,
	numbers=left,numberblanklines=true,numberstyle=\scriptsize\color{mylistingnclr}]
> ./cross_section.py -s 0 -e 10000 -h 100 -v 100 -d vs \
  -c cvms -a d -o 34,-118 -f 35,-117 -i image.png
\end{lstlisting}


\begin{figure*}[ht!]
	\centering
    \begin{subfigure}[b]{0.45\textwidth}
%    	\includegraphics[width=\textwidth]{cross-section}
		\fbox{\begin{minipage}{\textwidth}\textcolor{red}{PENDING}\vspace{2in}\end{minipage}}
        \caption{Cross section example of CVM-S showing \vs{} profiles in the Los Angeles basin produced with the \texttt{cross\_section.py} utility.}
        \label{fig:cross.section}
    \end{subfigure}%
	\qquad
	\begin{subfigure}[b]{0.45\textwidth}
%    	\includegraphics[width=\textwidth]{h-slize}
		\fbox{\begin{minipage}{\textwidth}\textcolor{red}{PENDING}\vspace{2in}\end{minipage}}
        \caption{Horizontal slice of CVM-S showing surface \vs{} on the Los Angeles basin produced with the \texttt{horizontal\_slice.py} utility.}
        \label{fig:h.slice}
    \end{subfigure}%
    \vspace{2ex}
    \begin{subfigure}[b]{0.45\textwidth}
%    	\includegraphics[width=\textwidth]{vs-thirty}
		\fbox{\begin{minipage}{\textwidth}\textcolor{red}{PENDING}\vspace{2in}\end{minipage}}
        \caption{Surface \vsthirty{} map of the Los Angeles produced with the \texttt{vs30.py} utility.}
        \label{fig:vs.thirty}
    \end{subfigure}%
	\qquad
	\begin{subfigure}[b]{0.45\textwidth}
%    	\includegraphics[width=\textwidth]{basin-depth}
		\fbox{\begin{minipage}{\textwidth}\textcolor{red}{PENDING}\vspace{2in}\end{minipage}}
        \caption{Depth of the Los Angeles basin at \vseq{1000} produced with the \texttt{z10.py} utility.}
        \label{fig:basin.depth}
    \end{subfigure}%
    \caption{Examples of the images obtained using the plotting utilities available in UCVM}
    \label{fig:plot.samples}
\end{figure*}

\subsection{The \textup{\texttt{horizontal\_slice.py}} utility}

This utility plots a horizontal slice of a model given two bounding latitude and longitude points, the depth at which the slice is to be taken, and other additional parameters. The output can be either an image or a text file containing the extracted data points from the model. The following example plots a horizontal slice at zero depth (surface) with... \textcolor{red}{give an explanation that matches the figure}. The resulting plot is shown in Figure \ref{fig:h.slice}.

\begin{lstlisting}[
	frame=none,
	basewidth={0.45em,0.4em},
	basicstyle=\ttfamily\footnotesize,breaklines=true,
	linewidth=0.98\columnwidth,xleftmargin=0.07\columnwidth,
	numbers=left,numberblanklines=true,numberstyle=\scriptsize\color{mylistingnclr}]
> ./horizontal_slice.py -b 34,-118 -u 35,-117 -s 0.01 \
  -e 0 -d vs -c cvms -a d -i image.png
\end{lstlisting}

\subsection{The \textup{\texttt{vs30.py}} utility}

This utility generates a \vsthirty{} map (i.e., the average \vs{} over the top 30 m) of a model given two bounding latitude and longitude points, along with a few other settings. As with the previous utilities, the output of this command can be either an image or a text file. The following example plots the \vsthirty map obtained from... \textcolor{red}{give an explanation that matches the figure}. Figure \ref{fig:vs.thirty} shows the result.

\begin{lstlisting}[
	frame=none,
	basewidth={0.45em,0.4em},
	basicstyle=\ttfamily\footnotesize,breaklines=true,
	linewidth=0.98\columnwidth,xleftmargin=0.07\columnwidth,
	numbers=left,numberblanklines=true,numberstyle=\scriptsize\color{mylistingnclr}]
> ./vs30.py -b 34,-118 -u 35,-117 -s 0.01 -z 0.1 -c cvms 
  -a d -i image.png
\end{lstlisting}

\subsection{The \textup{\texttt{z10.py}} and \textup{\texttt{z25.py}} utilities}

Both these utilities are intended to plot the depth of basins in a given model. The \texttt{z10.py} utility generates an isosurface map for the depths at which \vseq{1000}. The equivalent \texttt{z25.py} utility perfomrs the same operation but for the depth at which \vseq{2500} is reached. The output in either case, as with the previous plotting utilities, can be an image or a text file with the corresponding data points within a box defined by two bounding latitude and longitude coordinates. The following example plots the depth of the... \textcolor{red}{describe example}, and the resulting image is shown in Figure \ref{fig:basin.depth}.

