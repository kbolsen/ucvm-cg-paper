
\section{Introduction}
\label{sec:introduction}

The quantitative understanding of the physical world is an essential goal of geoscience research. We use mathematical abstractions to represent the behavior of systems under static and dynamic conditions and properties such as density and elastic moduli to characterize the capacity of materials to absorbe or transmit forces in stationary and transient processes. In seismology and geophysics, our understanding of physical phenomena associated with earthquakes, their genesis, and their effects depends on our knowledge and representation of the geometry and material properties of the Earth's structure and our capacity to represent the mechanical characteristics of the rupture process and subsequent seismic wave propagation. We use stress conditions and dynamic rupture models to describe faulting processes and seismic velocity and attenuation models along with wave propagation principles to determine the characteristics of the ground motion.

Initial stress models and seismic velocity models are therefore key inputs to earthquake ground motion simulation. We are interested in how seismic velocity models are built and made available to geoscientists, and in particular, how these models can help advance physics-based earthquake science. We utilize modeling approaches based on deterministic numerical techniques---such as finite element, finite difference, or spectral element methods---to simulate ground motion in ways that incorporate the physics of earthquake processes explicitly. That is, methods that explicitly solve the associated wave propagation problem. The use of physics-based earthquake simulation has increased over the last two decades thanks to the growth---in capacity and availability---of high-performance computing (HPC) facilities and applications \citep[e.g.,][]{Aagaard_2008_BSSA2, Olsen_2009_GRL, Bielak_2010_GJI, Cui_2010_Proc}. These simulations have focused on applications of great impact in seismology and earthquake engineering, such as the assessment of regional seismic hazard \citep[e.g.,][]{Graves_2011_PAG}.

Recent simulations have highlighted the impact of velocity models on the accuracy of simulation results \citep[e.g.,][]{Taborda_2014_BSSA}. Numerous seismic velocity models have been built to capture specific regional or local structures for use in simulations \citep[e.g.,][]{Frankel_1992_BSSA, Brocher_2008_BSSA, Graves_2008_BSSA}. The need for common models to be shared between investigators led to the concept of community velocity models (CVMs). CVMs are region-specific seismic velocity models that have been developed, maintained, advanced and used by a community of interested investigators. For example, CVMs have been developed for southern and northern California, Utah, and the central United States \citet{Kohler_2003_BSSA}, \citet{Suss_2003_JGR}, \citet{Brocher_2006_Proc}, \citet{Magistrale_2006_Tech}, \citet{Plesch_2011_SCEC}, and \citet{RamirezGuzman_2012_BSSA}. 

CVMs have been typically distributed in the form of datasets or collections of files, or in the form of computer programs that can dynamically operate on these datasets and files to provide information about the geometry and material properties of the crust in a particular region. However, these datasets and computer programs have not always been designed carefully from a software development perspective. In addition, recent advances in earthquake simulations, powered by the increasing capability of supercomputers, have increased significantly the computational demand placed on CVMs as input to these simulations.

This paper presents the Unified Community Velocity Model (UCVM), a software framework designed to provide standardized and computationally efficient access to seismic velocity models, developed and maintained by the Southern California Earthquake Center (SCEC). UCVM is a collection of software tools and application programming interfaces (APIs) that facilitate the access to the material properties stored in CVMs. Although UCVM was conceived as a tool to aid physics-based earthquake ground-motion simulation and regional seismic hazard assessment, it can be and has been used in other geoscience and engineering applications. Here, we describe the development of UCVM and its various software components and features, including some targeted for use in high-performance parallel computers, outline the velocity models available in UCVM and the process of integrating new CVMs, and present examples of recent applications of UCVM tools in geoscience and earthquake engineering research.

