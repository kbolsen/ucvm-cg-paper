
\section{Introduction}
\label{sec:introduction}

The quantitative understanding of the physical world is an essential goal of geoscience research. We use mathematical abstractions to represent the behavior of systems in static and dynamic states; and properties such as density and elastic moduli to characterize the response of materials to absorbe or transmit forces. In seismology and geophysics, our understanding of earthquakes and their impacts, depends in part on our knowledge and representation of the geometry and properties of seismically-active faults and of the Earth's crust itself. On the one hand, we use fault and stress models to describe the geometry and conditions that facilitate the initiation and dominate the propagation of an earthquake's rupture process; and, on the other hand, we use seismic velocity and attenuation models to describe how the geologic structure of the crust, and the mechanical properties of the materials in geologic structures, basins and sedimentary deposits controls the characteristics of seismic waves and the resulting ground motion generated by earthquakes.

We are interested in how seismic velocity models are managed by and distributed to scientists, with emphasis on simulation applications. Source models and seismic velocity models are the basic inputs to physics-based earthquake simulation. Physics-based earthquake simulation uses numerical methods and models that incorporate, explicitly, the solution of wave propagation problems through crustal structures, with specific applications in the assessment of the expected levels of ground motion in a region prone to earthquakes for use in seismic hazard analysis. During the last two decades, various seismic velocity models have been made available to the geophysics and seismology research community (e.g., add references). This gave way to the conception of community velocity models (CVMs). CVMs are seismic velocity models that have been developed, maintained, and advanced by a community of interested investigators who provide open-access to models and the information available therein (e.g., add references). 

CVMs have been typically distributed in the form of static datasets or computer programs that dynamically operate on those datasets to provide information about the geometry and material properties of the crust in a particular region. However, these datasets and computer programs have not always been thought carefully from a computational perspective. In addition, recent advances in earthquake simulations, powered by the increasing capability of supercomputers (e.g., add ref), have increased significantly the computational demand placed on CVMs as input to these simulations. 

This paper presents the Unified Community Velocity Model (UCVM), a software framework designed to provide standard and computationally efficient access to seismic velocity models. UCVM was developed at the Southern California Earthquake Center (SCEC). It is a collection of software tools and application programming interfaces (APIs) that facilitate the access to the material properties stored in CVMs, with specific applications in physics-based earthquake ground-motion simulation and regional seismic hazard assessments. Here, we describe the development of the UCVM and its various software components, including features for dedicated use in high-performance parallel computers, and present examples of recent applications of UCVM tools in earthquake research.


% ========================================================================================


%, that is, material models that provide information about the geometry and composition of the Earth's crust and the geomechanical properties of ---to . We focus on the latter and the means available to store 

% Materialized models, that is,  are one of the various forms available for representing media. In today's computer-driven environment, 3D models are typically distributed in the form of discrete datasets or 


% ========================================================================================


% DAVID 2013 MATERIAL
% ===================

%Three-dimensional (3D) seismic velocity models provide fundamental input data to ground motion simulations, in the form of structured or unstructured meshes or grids. Numerous models are available for California, as well as for other parts of the United States and Europe, but models do not share a common interface. Being able to interact with these models in a standardized way is critical in order to configure and run 3D ground motion simulations. The Unified Community Velocity Model (UCVM) software, developed by researchers at the Southern California Earthquake Center (SCEC), is an open source framework designed to provide a cohesive way to interact with seismic velocity models. 
%
%We describe the several ways in which we have improved the UCVM software over the last year. We have simplified the UCVM installation process by automating the installation of various community codebases, improving the ease of use.. We discuss how UCVM software was used to build velocity meshes for high-frequency (4Hz) deterministic 3D wave propagation simulations, and how the UCVM framework interacts with other open source resources, such as NetCDF file formats for visualization.
%
%The UCVM software uses a layered software architecture that transparently converts geographic coordinates to the coordinate systems used by the underlying velocity models and supports inclusion of a configurable near-surface geotechnical layer, while interacting with the velocity model codes through their existing software interfaces. No changes to the velocity model codes are required.
%
%Our recent UCVM installation improvements bundle UCVM with a setup script, written in Python, which guides users through the process that installs the UCVM software along with all the user-selectable velocity models. Each velocity model is converted into a standardized (configure, make, make install) format that is easily downloaded and installed via the script. UCVM is often run in specialized high performance computing (HPC) environments, so we have included checks during the installation process to alert users about potential conflicts.
%
%We also describe how UCVM can create an octree-based database representation of a velocity model which can be directly queried by 3D wave propagation simulation codes using the open source etree library. We will discuss how this approach was used to create an etree for a 4-Hz Chino Hills simulation.
%
%Finally, we show how the UCVM software can integrate NetCDF utility code to produce 3D velocity model files compatible with open source NetCDF data viewers. This demonstrates that UCVM can generate meshes from any compatible community velocity model and that the resulting models can be visualized without the need for complex secondary tools. This illustrates how developers can easily write tools that can convert data from one format to another using the UCVM API.

% PATRICK 2011 MATERIAL
% =====================

%The SCEC Unified California Velocity Model (UCVM) is a software framework for a state-wide California velocity model. UCVM provides researchers with two new capabilities: (1) the ability to query Vp, Vs, and density from any standard regional California velocity model through a uniform interface, and (2) the ability to combine multiple velocity models into a single state-wide model. These features are crucial in order to support large-scale ground motion simulations and to facilitate improvements in the underlying velocity models.
%
%UCVM provides integrated support for the following standard velocity models: SCEC CVM-H, SCEC CVM-S and the CVM-SI variant, USGS Bay Area (cencalvm), Lin-Thurber Statewide, and other smaller regional models. New models may be easily incorporated as they become available. Two query interfaces are provided: a Linux command line program, and a C application programming interface (API). The C API query interface is simple, fully independent of any specific model, and MPI-friendly. Input coordinates are geographic longitude/latitude and the vertical coordinate may be either depth or elevation. Output parameters include Vp, Vs, and density along with the identity of the model from which these material properties were obtained.
%In addition to access to the standard models, UCVM also includes a high resolution statewide digital elevation model, Vs30 map, and an optional near-surface geo-technical layer (GTL) based on Ely’s Vs30-derived GTL. The elevation and Vs30 information is bundled along with the returned Vp,Vs velocities and density, so that all relevant information is retrieved with a single query. When the GTL is enabled, it is blended with the underlying crustal velocity models along a configurable transition depth range with an interpolation function. 
%
%Multiple, possibly overlapping, regional velocity models may be combined together into a single state-wide model. This is accomplished by tiling the regional models on top of one another in three dimensions in a researcher-specified order. No reconciliation is performed within overlapping model regions, although a post-processing tool is provided to perform a simple numerical smoothing. Lastly, a 3D region from a combined model may be extracted and exported into a CVM-Etree. This etree may then be queried by UCVM much like a standard velocity model but with less overhead and generally better performance due to the efficiency of the etree data structure.

% PATRICK 2010 MATERIAL
% =====================

%The SCEC Community Velocity Model Toolkit (CVM-T) enables earthquake modelers to quickly build, visualize, and validate large-scale meshes using SCEC CVM-H or CVM-4. CVM-T is comprised of three main components: (1) an updated community velocity model for Southern California, (2) tools for extracting meshes from this model and visualizing them, and (3) an automated test framework for evaluating new releases of CVMs using SCEC’s AWP-ODC forward wave propagation software and one, or more, ground motion goodness of fit (GoF) algorithms.
%
%CVM-T is designed to help SCEC modelers build large-scale velocity meshes by extracting material properties from an extended version of Harvard University's Community Velocity Model (CVM-H). The CVM-T software provides a highly-scalable interface to CVM-H 6.2 (and later) voxets. Along with an improved interface to CVM-H material properties, the CVM-T software adds a geotechnical layer (GTL) to CVM-H 6.2+ based on Ely’s Vs30-derived GTL. The initial release of CVM-T also extends the coverage region for CVM-H 6.2 with a Hadley-Kanamori 1D background. Smoothing is performed within the transition boundary between the core model and the 1D background. The user interface now includes a C API that allows applications to query the model either by elevation or depth.
%
%The Extraction and Visualization Tools (EVT) include a parallelized 3D mesh generator which can quickly generate meshes (consisting of Vp, Vs, and density) from either CVM-H or CVM-4 with over 100 billion points. Python plotting scripts can be employed to plot horizontal or profile slices from existing meshes or directly from either CVM.
%
%The Automated Test Framework (ATF) is a system for quantitatively evaluating new versions of CVM-H and ensuring that the model improves against prior versions. The ATF employs the CruiseControl build and test framework to run an AWP-ODC simulation for the 2008 Chino Hills event (Mw = 5.39) and perform a goodness of fit statistics calculation on the generated synthetic and recorded observed seismograms using the GoF algorithm, based on comparison of synthetic peak amplitudes to observed peak amplitudes, used in the SCEC Broadband platform. CVM-T produced plots include comparisons of synthetic and observed seismograms, plots of bias versus period, and spatial plots of the pseudo-AA bias over the entire region.

