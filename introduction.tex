
\section{Introduction}
\label{sec:introduction}

The quantitative understanding of the physical world is an essential goal of geoscience research. We use mathematical abstractions to represent the behavior of systems under static and dynamic conditions; and properties such as density and elastic moduli to characterize the capacity of materials to absorbe or transmit forces in stationary and transient processes. In seismology and geophysics, our understanding of physical phenomena associated to earthquakes, their genesis, and effects, depends in a good measure on our knowledge and accurate representation of the geometry and material properties of the Earth's structure, as well as on our capacity to represent the mechanical characteristics of the rupture process that takes place when a seismic fault breaks and the subsequent seismic wave propagation problem. For the former case, we use stress conditions and dynamic rupture models to describe the faulting process. On the other hand, for the latter case, we use seismic velocity and attenuation models to describe the geologic structure of the mantle and crust, and the mechanical properties of the materials in the different geologic units that compose these structures to describe the propagation of seismic waves through basins and sedimentary deposits, and thus determine the characteristics of the ground motion.

Source models and seismic velocity models are therefore the basic input to earthquake simulation. We are interested on how seismic velocity models are built and made available to geoscientists, and in particular, on how these models can help advance physics-based earthquake simulation. We call physics-based earthquake simulation, the modeling approaches that use deterministic numerical techniques---such as the finite element, finite difference, or spectral element methods---to simulate the ground motion in ways that incorporate the physics of earthquake processes explicitly. That is, methods that explicitly solve the associated wave propagation problem. The use of physics-based earthquake simulation has increased considerably over the last two decades thanks to the growth---in capacity and availability---of high-performance computing (HPC) facilities and applications \citep[e.g.,][]{Aagaard_2008_BSSA2, Olsen_2009_GRL, Bielak_2010_GJI, Cui_2010_Proc}. These simulations have specific applications of great impact in seismology and earthquake engineering in aspects such as the assessment of regional seismic hazard \citep[e.g.,][]{Graves_2011_PAG}.

Recent simulations have highlighted the importance of velocity models in the accuracy of simulation results \citep[e.g.,][]{Taborda_2014_BSSA}. Numerous seismic velocity models have been built for specific regional or local structures and used in particular simulations over the years \citep[e.g.,][]{Frankel_1992_BSSA, Brocher_2008_BSSA, Graves_2008_BSSA}. The need for these models in simulation gave way to the conception of the community velocity models (CVMs). CVMs are seismic velocity models that have been developed, maintained, advanced and used by a community of interested investigators. Some examples of CVMs for the regions of southern and northern California, Utah, and the central United States are those models developed by \citet{Kohler_2003_BSSA, Brocher_2006_Proc, Magistrale_2006_Tech} and \citet{RamirezGuzman_2012_BSSA}. 

CVMs have been typically distributed in the form of datasets or collections of files, or in the form of computer programs that can dynamically operate on these datasets and files to provide information about the geometry and material properties of the crust in a particular region. However, these datasets and computer programs have not always been thought carefully from a computational perspective. In addition, recent advances in earthquake simulations, powered by the increasing capability of supercomputers, have increased significantly the computational demand placed on CVMs as input to these simulations. 

This paper presents the Unified Community Velocity Model (UCVM), a software framework designed to provide standardized and computationally efficient access to seismic velocity models, developed and maintained by the Southern California Earthquake Center (SCEC). UCVM is a collection of software tools and application programming interfaces (APIs) that facilitate the access to the material properties stored in CVMs. Although UCVM was conceived as a tool to aid physics-based earthquake ground-motion simulation and regional seismic hazard assessment, it can be used in other geosciences and engineering applications. Here, we describe the development of UCVM and its various software components, including features for use in high-performance parallel computers, and present examples of recent applications of UCVM tools in earthquake research.

